% !TEX TS-program = pdflatex
% !TEX encoding = UTF-8 Unicode

% This is a simple template for a LaTeX document using the "article" class.
% See "book", "report", "letter" for other types of document.

\documentclass[12pt]{article} % use larger type; default would be 10pt

\usepackage[utf8]{inputenc} % set input encoding (not needed with XeLaTeX)

%%% Examples of Article customizations
% These packages are optional, depending whether you want the features they provide.
% See the LaTeX Companion or other references for full information.

%%% PAGE DIMENSIONS
\usepackage{geometry} % to change the page dimensions
\geometry{a4paper} % or letterpaper (US) or a5paper or....
% \geometry{margin=2in} % for example, change the margins to 2 inches all round
% \geometry{landscape} % set up the page for landscape
%   read geometry.pdf for detailed page layout information

\usepackage{graphicx} % support the \includegraphics command and options

% \usepackage[parfill]{parskip} % Activate to begin paragraphs with an empty line rather than an indent

%%% PACKAGES
\usepackage{booktabs} % for much better looking tables
\usepackage{array} % for better arrays (eg matrices) in maths
\usepackage{paralist} % very flexible & customisable lists (eg. enumerate/itemize, etc.)
\usepackage{verbatim} % adds environment for commenting out blocks of text & for better verbatim
\usepackage{subfig} % make it possible to include more than one captioned figure/table in a single float
% These packages are all incorporated in the memoir class to one degree or another...

%%% ToC (table of contents) APPEARANCE
\usepackage[nottoc,notlof,notlot]{tocbibind} % Put the bibliography in the ToC
\usepackage[titles,subfigure]{tocloft} % Alter the style of the Table of Contents
\renewcommand{\cftsecfont}{\rmfamily\mdseries\upshape}
\renewcommand{\cftsecpagefont}{\rmfamily\mdseries\upshape} % No bold!

%%% END Article customizations

%%% The "real" document content comes below...

\title{CS 5540 Project 1}
\author{Arjun Biddanda (aab227) and Muhammad Khan (mhk98)}
%\date{} % Activate to display a given date or no date (if empty),
         % otherwise the current date is printed 

\begin{document}
\maketitle

\section{Determining Neuron Sensitivity}
\subsection{Stimulus Class}
The $\tt{Stimulus}$ class is used to give an object definition in Java for each particular stimulus. We define the stimulus as a Java $\tt{enum}$ (a strict or-type enumerative declaration), so there are no fields that we have to extend for each stimulus. The stimuli are used as defining pieces of data with regards to our construction of the $\tt{Neuron}$ class.

\subsection{Neuron Class}
This is a class that we developed in order to accurately represent a neuron. The defining feature of the class is the map that is representative of the spike trains that are recorded for each stimulus. We define the map as $spiketimes$, which is a hashmap which takes a stimulus as a key and outputs a double array detailing all of the spike times for that particular stimulus. 
%reword the algorithm used here a bit
\subsubsection{Sensitive Stimuli}
This is a simple method to compute (roughly) whether a neuron is sensitive to a a particular stimulus. We first create an outer loop to iterate through the list of all possible stimuli, where we choose a stimulus $s$. Within that outer loop, we have two inner loops to iterate through the different spike time trials for $s$. We have an average counter $avg$ which maintains the average spike number from all of the trials that we have gone through for that particular stimulus. 

At the termination of the inner loop we have a check to decide if the average number of spikes is high enough for us to determine that it is significant. We have set the "cutoff" point for the number of spikes to be an average of 30 for a given stimulus $s$. This was arbitrarily decided from looking at the training data that we had divided our set into. 

After applying this to the set of all neurons presented to us, we would theoretically have a length 35 array of variable lists containing the significant stimuli to the neuron at whichever index $i$ we were at along the array. 

\subsection{DataParser}
In order to use the data provided within the text file, we had to develop a  parser which will create all of the neurons that we require. The parser is not optimized for speed in any way, because it just sequentially goes through the text file and every time that it hits a line that begins with the character 'N' it just remembers to create a new Neuron object and add the old one to the accumulator list. At the termination of the algorithm we will have an array filled with 35 neuron objects (because that is how large our data set is).

\section{Creating a Response Space}
\subsection{Initial Assumptions}
Our initial method by which to calculate significant stimuli was quite naive really. Since we are going purely by the number of spikes that are generated, the spread of the data is not considered whatsoever, and that is very dangerousin our interpretation of this data. For instance, if we are given spike trains $A$ and $B$, where $A$ has a large number of spikes at the ends of the time intervals, and $B$ has an equally large number of spikes in the middle. According to our first model, they would appropriately be assigned to the fact that they have the same significant stimuli. 


\subsection{ResponseSpace}



More text.

\end{document}
